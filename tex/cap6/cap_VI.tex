\chapter{Considera��es Finais}
\label{cap:estudoCaso}

\section{Conclus�o}
\label{sec:conclusao}
A motiva��o para o desenvolvimento do projeto, al�m de servir como pr�tica do assuntos estudados e abordados no curso de p�s-gradua��o. Gera a possibilidade de se criar um produto que pode ser lan�ado no mercado com foco nos centros automotivos de pequeno e m�dio porte e profissionais liberais com um custo mais acess�vel para esse nicho. Visando sempre em primeiro lugar a melhor experi�ncia para o usu�rio ao utilizar o sistema, a simplicidade nos fluxos do sistema, telas limpas e com poucas informa��es fecham a fase de estudos te�ricos e colocam em pr�tica tudo o que foi aprendido.
\subsection{Contribui��o}
\label{subsec:contribuicao}
Neste trabalho foi poss�vel desenvolver um  sistema de controle de or�amentos e ordens de servi�o para centros automotivos de pequenos e m�dio porte e profissionais liberais.
\begin{itemize}
\item
Foi estudado o referencial te�rico para o desenvolvimento de um software;
\item
Recolhido os requisitos funcionais e n�o funcionais;
\item
Foi realizado o projeto de Banco de Dados;
\item
Implementado um sistema de software denominado \nomeTrabalho;
\item 
Foi criado o mapa mental do sistema
\item
Um app foi publicado na google play, que apoia o uso do cliente web pelo centro automotivo.
\end{itemize}

\section{Trabalhos Futuros}
\label{sec:trabalhosFuturos}
Para trabalhos futuros, na vers�o 2.0.0 ficam:
\begin{enumerate}
\item Gera��o de relat�rios;
\item Melhorias na usabilidade do cliente web e cliente mobile de acordo com o retorno dos usu�rios;
\item Implementa��o de grupos de acesso com diferentes perfis;
\item Integra��o com software de controle de estoque de produtos;
\item Integra��o com software de controle financeiro;
\end{enumerate} 