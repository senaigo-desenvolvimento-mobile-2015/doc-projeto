\chaves{Mobile, Web, REST, Restful, Aurelia, JavaScript, NodeJS.}

\begin{resumo}
\textit 
O Or�amento WEB � uma suite multi-plataforma, que integra a plataforma
Web com a mobile atrav�s de uma arquitetura
\textit{RestFul}, provendo assim, servi�os \textit{rest} na web. Trata-se de um
sistema para cria��o e acompanhamento de or�amentos e ordens de servi�o (O.S)
\textit{on-line} de forma simples, r�pida e direta. Para isso, foi criado uma \textit{API}, que disponibiliza os servi�os, regras de neg�cio e acesso aos dados,
desenvolvida utilizando \textit{JavaScript} com NodeJS e banco de dados
PostgreSQL. Um cliente Web, desenvolvido em Html5, Css3 e \textit{JavaScript}
utilizando o \textit{framework} Aurelia em sua vers�o $1.0.0$. O cliente Web tem
o objetivo, dentre outros, de ser o console administrativo do
sistema, atrav�s do uso da \textit{API} criada. Um cliente m�vel desenvolvido em Android nativo, que
possibilitar� ao cliente da empresa, acompanhar o \textit{status} da Ordem de
Servi�o do seu ve�culo em um centro automotivo. O projeto de software criado
demonstra que � poss�vel criar uma estrutura de integra��o entre as diferentes
plataformas (Web e Mobile) de maneira homog�nea e transparente.
\end{resumo}